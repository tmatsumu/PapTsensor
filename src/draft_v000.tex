\documentclass[12pt]{amsart}
\usepackage{geometry} % see geometry.pdf on how to lay out the page. There's lots.
\geometry{a4paper} % or letter or a5paper or ... etc
% \geometry{landscape} % rotated page geometry

% See the ``Article customise'' template for come common customisations

\title{Remote sensing measurement of a cryogenic temperature using a levitating NdFeB rotor of a superconducting magnetic bearing}
\author{}
\date{} % delete this line to display the current date

%%% BEGIN DOCUMENT
\begin{document}

\maketitle
\tableofcontents

\section{Introduction}
This is git commit test from Yuki Sakurai.
\begin{itemize}
\item Temperature monitoring at cryogenic temperature
\item remote sensing
\item bolometer or some other methods
\item SMB is a good tool to achieve the continuous motion at the cryogenic temperature
\item motivation, we want to know the rotor temperature when it is spinning
\item We demonstrate the remote sensing temperature measurement method using temperature dependence of the magnetisation
\item each chapter describe what
\end{itemize}

\section{Method}

\begin{itemize}
\item model
\item calibration method
\item calibration sequence
\end{itemize}


\begin{eqnarray}
	B(T,\vec{r},\vec{r_0}) = B_r(T) G(\vec{r},\vec{r_0})
\end{eqnarray}

\section{Experimental setup}
\begin{itemize}
\item inside including the SMB
\item hall sensor and temperature sensor
\item temperature sensor accuracy
\item DAQ
\end{itemize}

\section{Results}
\subsection{Temperature dependence of the remnance}

\subsection{Thermal model}

\subsection{Estimation of the rotor temperature}

\section{Discussions}
\subsection{Bias due to the change of the levitation height}

\subsection{Non-uniformity}

\section*{Acknowledgement}

\end{document}
