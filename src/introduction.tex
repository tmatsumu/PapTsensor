\section{Introduction}

Measurements of temperature is one of the basic techniques to support any physics experiment.
In particular, a cryogenically cooled experiment requires extra effort to measure various temperature stage.
The standard technique is to employ a temperature dependent resistor or diode.
In this way, we read the voltage across each temperature dependent electrical component given a constant current bias.

This type of a temperature sensor relies on a conduction between the temperature sensor and the temperature stage to be measured.
In typical applications, a temperature sensor is mounted on a temperature stage to be measured by a screw or glued, and one has to ensure that the sensor and the stage is isothermal.

This technique does not work when an object of our interest is levitating and there is no path of conduction.
A superconducting magnetic bearing is a passively levitating bearing, which consists of an array of high temperature superconducting (HTS) tile and a permanent magnet.
When the HTS tiles are cooled below critical temperature with a presence of the magnetic field, the fluxed is trapped.
As a result, the rotor magnet levitates as long as the stator HTS is maintained below its critical temperature.
When the rotor magnet has an axial symmetry, the rotor magnet does not only levitate but also it rotates about the axis of magnetic symmetry.

This SMB is a contactless, and thus the coefficient of friction between the rotor and the stator is as low as $*10^{-4}*$ or less.
Due to this nature, this technology is employed to various applications, including a flywheel energy storage, a dust-free conveyer, a maglev train, and a polarization modulator in astrophysical observations.
While the attractive feature of SMB is achieved by the fact that the bearing is levitating and contactless, this feature makes difficult to measure the rotor temperature.

The rotor of the SMB can be used from 4K to room temperature.
At the room temperature range, the IR remote sensing temperature sensor can be used to measure the levitating and rotating temperature.
As the rotor temperature decreases, the need of the detector becomes complicated.
At 4K, millimeter wave bolometer.
At * K, Terahertz range detector.

In this paper, we present the demonstration of the remote temperature sensing technique at cryogenic temperature (* K - * K) using a temperature dependent magnetic remnance of a rotor magnet. In Section~\ref{}, we describe the concept. In Section~\ref{}, we describe the experimental setup and the results. Finally in Section~\ref{}, we discuss the systematic bias and potential applications.

\begin{itemize}
\item Temperature monitoring at cryogenic temperature
\item remote sensing
\item bolometer or some other methods
\item SMB is a good tool to achieve the continuous motion at the cryogenic temperature
\item motivation, we want to know the rotor temperature when it is spinning
\item We demonstrate the remote sensing temperature measurement method using temperature dependence of the magnetisation
\item each chapter describe what
\end{itemize}
