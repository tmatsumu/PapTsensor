\section{Introduction}

A measurement of temperature is one of the basic techniques to support any physics experiment. 
In particular, an experiment which involves a cryogenic temperature requires extra cares in the measurements of various temperatures.

A standard technique to measure the cryogenic temperature is to employ a thermometer that uses a temperature dependent resistor or diode. 
In this way, we read the voltage across a thermometer given a constant electrical current bias. 
This type of a thermometer relies on a conduction between the temperature sensor and an object of interest. 
In typical applications, a temperature sensor is mounted on the object of interest by a screw or a glue in order to ensure that the two are isothermal. 

This technique does not work when an object of interest is levitating, and thus there is no path of conduction. 
A superconducting magnetic bearing (SMB) falls into such a category. 
SMB is a passively levitating bearing, which consists of an array of high temperature superconducting (HTS) tile and a permanent magnet. 
When the HTS tiles are cooled below its critical temperature with a presence of the external magnetic field, the fluxed is trapped. 
As a result, the rotor magnet, i.e. a source of the external magnetic field, becomes stable in all degrees of freedom. 
This magnet levitates against to the gravity without any mechanical support as long as the stator HTS is maintained below its critical temperature. 
When the rotor magnet has an axial symmetry, the rotor magnet does not only levitate but also it rotates about the axis of magnetic symmetry. 

This SMB is a contactless bearing, and thus the coefficient of friction between the rotor and the stator is as low as $*10^{-4}*$ or less. 
Due to this nature, this technology is employed in various applications, including a flywheel energy storage, a dust-free conveyer, a maglev train, and a polarization modulator in astrophysical observations.
While the attractive feature of SMB is achieved by the fact that the bearing is levitating without contact, this feature prevents to measure the rotor temperature via a thermometer the conventional technique. 

The rotor of the SMB can be used from 4~K to room temperature. 
The potential method to measure the rotor temperature is followings.
At the room temperature range, the IR remote sensing temperature sensor can be used to measure the levitating and rotating temperature. 
As the rotor temperature decreases, the need of the detector becomes complicated. 
At 4~K, millimeter wave bolometer. 
At *~K, Terahertz range detector. 

In this paper, we present the demonstration of the remote temperature sensing technique at cryogenic temperature (*$\sim$ *~K) using a temperature dependent magnetic remnance of a rotor magnet. 
This measurement technique does not inject any energy to the levitating and spinning rotor, and thus this is purely passive system. 
In Section~\ref{}, we describe the concept. In Section~\ref{}, we describe the experimental setup and the results. Finally in Section~\ref{}, we discuss the systematic bias and potential applications.

\begin{itemize}
\item Temperature monitoring at cryogenic temperature
\item remote sensing
\item bolometer or some other methods
\item SMB is a good tool to achieve the continuous motion at the cryogenic temperature
\item motivation, we want to know the rotor temperature when it is spinning
\item We demonstrate the remote sensing temperature measurement method using temperature dependence of the magnetisation
\item each chapter describe what
\end{itemize}
